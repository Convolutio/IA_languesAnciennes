\documentclass[12pt, a4, french]{report}
\usepackage[total={6.5in,10in}, top=0.8in, left=1in, includefoot]{geometry}
\usepackage[utf8]{inputenc}
\usepackage[T1]{fontenc}
\usepackage{babel, csquotes, xpatch}
% to install csquotes, xpatch and biblatex, download their zips on ctan.org
% and after (excepted for xpatch) copy the unzipped folders witht the .sty files
% in the <texmf-dist>/tex/latex/ repository (see tex-workshop logs to figure out
% the path of <texmf-dist>).
% For xpatch, read the README of its zip.
\usepackage[
    backend=biber,
    style=numeric
]{biblatex}
\addbibresource{biblio.bib}

\begin{document}
\begin{titlepage}
    \centering
    \vspace*{\fill}

    \huge\bfseries
    Les utilisations possibles de l'Intelligence Artificielle dans la linguistique historique
    
    \vspace*{1.5cm}
    \large 3 étudiants de CPBx
    
    \vspace*{\fill}
\end{titlepage}

\section{Résumé}
\section{Abstract}
\section{Remerciements}

\tableofcontents
\section{Table des figures}
\section{Notations}

\chapter{Introduction}
\textit{Mise en contexte pour arriver à la problématique, quel est le potentiel de l'intelligence artificielle dans la linguistique historique ? ... ?}

\chapter{La linguistique historique et l'Intelligence Artificielle}
\section{La linguistique historique}
\subsection{Introduction à la linguistique historique}
\textit{Définir ce qu'est la linguistique historique, ce qu'elle étudie, et les mots de vocabulaires que nous allons rencontrer tout au long du mémoire.}
\subsection{Les différents principes}
\textit{Évidemment cette science repose sur des concepts, allant des propriétés synchoniques des mots aux à leurs aspects diachroniques.}
\subsection{Les atouts de l'Intelligence Artificielle dans ce domaine}
\textit{La linguistique historique fait face à de nombreux problèmes récurrents (traiter une grande quantité de textes pour l'homme, remarquer des motifs dans ces documents historiques). Alors que ce travail pourrait être effectué par une machine, grâce à sa capacité à traiter un grand nombres de données, et à chercher des similarités dans ces données. Avant, de voir les tâches où l'Intelligence Artificielle peut intervenir, il est d'abord nécessaire de voir en détail la conception des ces IA.}

Résoudre des problèmes de Linguistique Historique avec un ordinateur nécessite de lui faire traiter
du contenu textuel devant être abstrait sur des terrains parmi ceux de la \textbf{phonétique}, de la
\textbf{sémantique}, de la \textbf{morphologie} ou encore de la \textbf{syntaxe}.\\
\textbf{\textit{Développper un exemple pour illustrer ces 4 niveaux d'abstractions}}

La réalisation de ces abstractions s'inscrit dans le Traitement Automatisé du Langage Naturel (TAL),
un domaine à cheval entre la Linguistique et l'Informatique. L'Intelligence Artificielle y occupe
une place centrale pour sa capacité à effectuer des approximations améliorables avec de l'entraînement.


\section{L'IA dans le Traitement Automatisé du Langage Naturel}
\subsection{Introduction à l'apprentissage automatique}
\textit{Qu'est ce qu'une intelligence artificielle ?\\
    Qu'est ce qu'un réseau de neurones ?\\
    Quel est le principe derrière l'apprentissage automatique ?\\
    Définition des apprentissages supervisés/non supervisés
    Définition de propagation avant.
    Définition rétro-propagation du gradient.
    Exemple de FFNN pour tâche de classification}



\subsection{Les pré-traitements nécessaires du texte.}
\textit{tokenisation + normalisation des données\\
    Vectorisation sémantique des mots (embeddings)\\
    Encodage d'embeddings (statique ou contextuelle)}

\subsection{Architectures neuronales utiles au TAL}
\textit{Quels sont les différents outils ?}\\
\textit{Suivant, comment les parties précédentes ont été traités, ou comment les parties futures seront discutées, cette partie pourrait ne pas être nécessaire. Sinon, elle regroupera l'idée de comment on passe de notre langue naturelle à celle de la machine, de passer aux mots à des vecteurs ? Quels traitements théoriques (théoriques pour ce distinguer de la pratique dans la partie future) doient être effectués sur les mots ? En fait cette partie fait référence aux chapitres 2 et 6 de Jurasky. Voir même le chapitre 9, en supprimant la sous partie précendente pour pouvoir parler directement des réseaux de neurones appliqués à la linguistique, en d'autres termes, des réseaux récurrents, des modèles séquentielles (encodeurs-décodeurs) avec l'attention, et des Transformers.}\\

\subsubsection{Réseaux de neurones récurrents}

\subsubsection{Transformeurs}


\chapter{Les contributions de l'IA dans la linguistique historique}
\textit{C'est la partie 'Related Work', elle discute des différents aspects où la linguistique historique s'applique, à travers différents modèles.}
\section{Restoration de documents anciens}
\section{Déchiffrement de langues anciennes}

\chapter{Étude du cas de l'application de l'IA pour la reconstruction des proto-formes d'une langue}
\textit{Ici, on se place dans un cas concret, pour montrer que ce n'est pas que de la théorie. En proposant une expérience.}
\section{État de l'art}
\subsection{Conceptualisation du problème}
\textit{Définir clairement le problème du titre, énoncer et justifier le choix de notre modèle réseaux de neurones et des différents outils appliqués. Voir s'il est possible de faire apparaitre plusieurs démarches, c'est à dire, une approche statistique et une approche neuronale (toujours pour renforcer et montrer le potentiel de l'IA).}

\subsection{Dernières solutions neuronales}
\textit{Solution supervisée + non supervisée}

\subsection{Limites d'applicabilité}
\textit{Expliquer en quoi le non-supervisé donne plus d'espoir
que le supervisé mais en quoi même cette approche présente des
limites.\\
Transition avec la problématique de l'article scientifique}
\section{Observation expériementale d'un limite d'applicabilité d'une approche}
\subsection{Méthode}
\subsection{Récupération de la base de données}
\textit{Il est fort possible que cette partie se regroupe avec la partie suivante, car il n'y aura pas grand chose à dire.}

\subsection{Potabiliser les données}
\textit{Le choix de "potabiliser", et non pas normaliser, est volontaire. En effet, la normalisation de nos données s'effectura dans un second temps dans les différentes démarches.}
\textit{Il reste ici quelques sous parties à détailler.}

\subsection{Analyse}
\subsection{Critiques}
\textit{Il reste ici quelques sous parties à détailler.}

\chapter{Conclusion}
\section{Synthèse}
\textit{Résume tout ce qui a été dit.}
\section{Les différentes limites posées aujourd'hui}
\textit{une partie des limites aura déjà été traitée dans le chapitre précédent. Cette sous partie se veut résumer ces limites, et aller dans les limites générales (voir acutelles) de l'IA dans  la linguisitique historique.}
\section{Les perspectives de l'IA dans la linguistique historique}
\textit{Ouverture, dépassement de certaines limites, évolution des modèles.}

\chapter{Références}
\section{Bibliographie}

\printbibliography

\end{document}
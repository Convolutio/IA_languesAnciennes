\documentclass[12pt, french]{report}
%\usepackage[total={6.5in,10in}, top=0.8in, left=1in, includefoot]{geometry}
\usepackage[a4paper, total={6.5in, 10in}, top=0.8in, left=1in,
    headheight=50pt,
    includefoot, includehead]{geometry}
\usepackage[utf8]{inputenc}
\usepackage[T1]{fontenc}
\usepackage{mathptmx} %Times font (contraint par le CPBx)

\usepackage{babel, csquotes, xpatch}
% to install csquotes, xpatch and biblatex, download their zips on ctan.org
% and after (excepted for xpatch) copy the unzipped folders witht the .sty files
% in the <texmf-dist>/tex/latex/ repository (see tex-workshop logs to figure out
% the path of <texmf-dist>).
% For xpatch, read the README of its zip.
\usepackage[
    backend=biber,
    style=numeric
]{biblatex}
\addbibresource{biblio.bib}

\usepackage{amsfonts}
\usepackage{fancyhdr} % Pour la mise en page des en-têtes

\begin{document}
\pagestyle{fancy}
\fancyhead{}
\fancyhead[R]{\nouppercase{\hfill\leftmark}\\}
\fancyhead[L]{\nouppercase{\rightmark\hfill}}
\fancyfoot[L]{Mémoire CPBx, 2023}
\begin{titlepage}
    \centering
    \vspace*{\fill}

    \huge\bfseries
    Les utilisations possibles de l'Intelligence Artificielle dans la linguistique historique
    
    \vspace*{1.5cm}
    \large 3 étudiants de CPBx
    
    \vspace*{\fill}
\end{titlepage}

\null
\newpage % Première page blanche pour la numérotation (contrainte de mise en page)
\section{Résumé}
\section{Abstract}
\section{Remerciements}

\tableofcontents
\section{Table des figures}
\section{Notations}

\chapter{Introduction}
\textit{Mise en contexte pour arriver à la problématique, quel est le potentiel de l'intelligence artificielle dans la linguistique historique ? ... ?}

\chapter{La linguistique historique et l'Intelligence Artificielle}
\section{La linguistique historique}
\subsection{Introduction à la linguistique historique}
\textit{Définir ce qu'est la linguistique historique, ce qu'elle étudie, et les mots de vocabulaires que nous allons rencontrer tout au long du mémoire.}
\subsection{Les différents principes}
\textit{Évidemment cette science repose sur des concepts, allant des propriétés synchoniques des mots aux à leurs aspects diachroniques.}
\subsection{Les atouts de l'Intelligence Artificielle dans ce domaine}
\textit{La linguistique historique fait face à de nombreux problèmes récurrents (traiter une grande quantité de textes pour l'homme, remarquer des motifs dans ces documents historiques). Alors que ce travail pourrait être effectué par une machine, grâce à sa capacité à traiter un grand nombres de données, et à chercher des similarités dans ces données. Avant, de voir les tâches où l'Intelligence Artificielle peut intervenir, il est d'abord nécessaire de voir en détail la conception des ces IA.}

Résoudre des problèmes de Linguistique Historique avec un ordinateur nécessite de lui faire traiter
du contenu textuel devant être abstrait sur des terrains parmi ceux de la \textbf{phonétique}, de la
\textbf{sémantique}, de la \textbf{morphologie} ou encore de la \textbf{syntaxe}.\\
\textbf{\textit{Développper un exemple pour illustrer ces 4 niveaux d'abstractions}}

La réalisation de ces abstractions s'inscrit dans le Traitement Automatisé du Langage Naturel (TAL),
un domaine à cheval entre la Linguistique et l'Informatique. L'Intelligence Artificielle y occupe
une place centrale pour sa capacité à effectuer des approximations améliorables avec de l'entraînement.


\section{L'IA dans le Traitement Automatisé du Langage Naturel}
\subsection{Introduction à l'apprentissage automatique}
\textit{Qu'est ce qu'une intelligence artificielle ?\\
    Qu'est ce qu'un réseau de neurones ?\\
    Quel est le principe derrière l'apprentissage automatique ?\\
    Définition des apprentissages supervisés/non supervisés
    Définition de propagation avant.
    Définition rétro-propagation du gradient.
    Exemple de FFNN pour tâche de classification}

De nombreux problèmes informatiques peuvent être résolus à travers la détermination
d'une fonction mathématique $f$ d'un $\mathbb{K}$-espace vectoriel $E$ vers un 
$\mathbb{K}$-espace $F$ (avec $\mathbb{K}$ correspondant à $\mathbb{R}$ ou
$\mathbb{C}$).\\
Ainsi, lorsqu'une fonction informatique conventionnelle effectue un traitement sur
une chaîne de caractères, une fonction $f$ a déjà été implicitement déterminée pour
réaliser la tâche.
La séquence de $n$ caractères encodés sous forme de bits forme un vecteur de l'espace
$\mathbb{R}^n$ et la chaîne renvoyée par $f$ est bien un élément d'un espace
$\mathbb{R}^{n'}$.

Il est aussi très fréquemment difficile -- voire impossible --
de poser une expression mathématique ou un algorithme pour répondre à certains 
problèmes. Dans ce cas là, $f$ est considérée comme hypothétique et on cherche à
l'approcher à partir d'un \textbf{modèle}, qu'on construit à partir des informations
qu'on dispose sur $f$, comme un ensemble de ses points 
$\{(x_k, y_k=f(x_k)), k \in S\}$.

\vspace{12pt}
Les \textbf{réseaux de neurones} sont d'excellents outils pour établir des modèles.
Mathématiquement, ce sont des compositions d'applications non-linéaires et linéaires
recevant un vecteur d'entrée représentant une donnée et sortant un vecteur de sortie
représentant un résultat dans un format cohérent avec le problème.\\
Le neurone artificiel le plus élémentaire effectue la \textbf{somme pondérée} des 
coefficients du vecteur d'entrée, puis calcule l'image de cette somme à travers
une \textbf{fonction d'activation}. La sortie du neurone est donc un réel ou un
complexe. Si on la note $y_j$, qu'on note $x$ le vecteur d'entrée dans $\mathbb{K}^n$,
$w_j$ le vecteur de \textbf{poids} associé au neurone et $\sigma$ sa fonction d'activation,
on a :
\begin{equation}
    y_j = \sigma(\sum_{i=0}^{n} w_{ij}x_i) = \sigma(<w_j, x>)
\end{equation}

\textit{introduire la matrice $W$ pour le calcul d'une couche\\
dire que les informations dans chaque coeff du vecteur x sont 
souvent abstraites et n'ont de sens que pour la machine (teaser vers sous-
section suivante) + leur importance est déterminée par les poids avec
l'entraînement\\ décrire l'entraînement (régression logistique+fct perte) \\
Exemple de la tâche de classification (connotation textuelle)\\
dire que $\sigma$ est un paramètre du réseau\\
on peut empiler plusieurs réseaux de neurones et plusieurs couches (paramètre
architecturale)}

\subsection{Les pré-traitements nécessaires du texte.}
\textit{tokenisation + normalisation des données\\
    Vectorisation sémantique des mots (embeddings)\\
    Encodage d'embeddings (statique ou contextuelle)}

\subsection{Architectures neuronales utiles au TAL}
\textit{Quels sont les différents outils ?}
\textit{Suivant, comment les parties précédentes ont été traités, ou comment les parties futures seront discutées, cette partie pourrait ne pas être nécessaire. Sinon, elle regroupera l'idée de comment on passe de notre langue naturelle à celle de la machine, de passer aux mots à des vecteurs ? Quels traitements théoriques (théoriques pour ce distinguer de la pratique dans la partie future) doient être effectués sur les mots ? En fait cette partie fait référence aux chapitres 2 et 6 de Jurasky. Voir même le chapitre 9, en supprimant la sous partie précendente pour pouvoir parler directement des réseaux de neurones appliqués à la linguistique, en d'autres termes, des réseaux récurrents, des modèles séquentielles (encodeurs-décodeurs) avec l'attention, et des Transformers.}\\

\subsubsection{Réseaux de neurones récurrents}

\subsubsection{Transformeurs}


\chapter{Les contributions de l'IA dans la linguistique historique}
\textit{C'est la partie 'Related Work', elle discute des différents aspects où la linguistique historique s'applique, à travers différents modèles.}
\section{Restoration de documents anciens}
\section{Déchiffrement de langues anciennes}

\chapter{Étude du cas de l'application de l'IA pour la reconstruction des proto-formes d'une langue}
\textit{Ici, on se place dans un cas concret, pour montrer que ce n'est pas que de la théorie. En proposant une expérience.}
\section{État de l'art}
\subsection{Conceptualisation du problème}
\textit{Définir clairement le problème du titre, énoncer et justifier le choix de notre modèle réseaux de neurones et des différents outils appliqués. Voir s'il est possible de faire apparaitre plusieurs démarches, c'est à dire, une approche statistique et une approche neuronale (toujours pour renforcer et montrer le potentiel de l'IA).}

\subsection{Dernières solutions neuronales}
\textit{Solution supervisée + non supervisée}

\subsection{Limites d'applicabilité}
\textit{Expliquer en quoi le non-supervisé donne plus d'espoir
que le supervisé mais en quoi même cette approche présente des
limites.\\
Transition avec la problématique de l'article scientifique}
\section{Observation expériementale d'une limite d'applicabilité d'une approche}
\subsection{Méthode}
\subsection{Récupération de la base de données}
\textit{Il est fort possible que cette partie se regroupe avec la partie suivante, car il n'y aura pas grand chose à dire.}

\subsection{Potabiliser les données}
\textit{Le choix de "potabiliser", et non pas normaliser, est volontaire. En effet, la normalisation de nos données s'effectura dans un second temps dans les différentes démarches.}
\textit{Il reste ici quelques sous parties à détailler.}

\subsection{Analyse}
\subsection{Critiques}
\textit{Il reste ici quelques sous parties à détailler.}

\chapter{Conclusion}
\section{Synthèse}
\textit{Résume tout ce qui a été dit.}
\section{Les différentes limites posées aujourd'hui}
\textit{une partie des limites aura déjà été traitée dans le chapitre précédent. Cette sous partie se veut résumer ces limites, et aller dans les limites générales (voir acutelles) de l'IA dans  la linguisitique historique.}
\section{Les perspectives de l'IA dans la linguistique historique}
\textit{Ouverture, dépassement de certaines limites, évolution des modèles.}

\chapter{Références}
\section{Bibliographie}

\printbibliography

\end{document}
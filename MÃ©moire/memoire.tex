\documentclass[12pt, letterpaper]{report}
\usepackage[total={6.5in,10in}, top=0.8in, left=1in, includefoot]{geometry}
\usepackage[utf8]{inputenc}
\usepackage[T1]{fontenc}
\usepackage[french]{babel}

\begin{document}
\begin{titlepage}
    \centering
    \vspace*{\fill}

    \huge\bfseries
    Les utilisations possibles de l'Intelligence Artificielle dans la linguistique historique
    
    \vspace*{1.5cm}
    \large 3 étudiants de CPBx
    
    \vspace*{\fill}
\end{titlepage}

\section{Résumé}
\section{Abstract}
\section{Remerciements}

\tableofcontents
\section{Table des figures}
\section{Notations}

\chapter{Introduction}
\textit{Mise en contexte pour arriver à la problématique, quel est le potentiel de l'intelligence artificielle dans la linguistique historique ? ... ?}

\chapter{La linguistique historique et l'intelligence artificielle}
\section{La linguistique historique}
\subsection{Introduction à la linguistique historique}
\textit{Définir ce qu'est la linguistique historique, ce qu'elle étudie, et les mots de vocabulaires que nous allons rencontrer tout au long du mémoire.}
\subsection{Les différents principes}
\textit{Évidemment cette science repose sur des concepts, allant des propriétés synchoniques des mots aux à leurs aspects diachroniques.}
\subsection{L'intégration de l'intelligence artificielle dans ce domaine}
\textit{La linguistique historique fait face à de nombreux problèmes récurrents (traiter une grande quantité de textes pour l'homme, remarquer des motifs dans ces documents historiques). Alors que ce travail pourrait être effectué par une machine, grâce à sa capacité à traiter un grand nombres de données, et à chercher des similarités dans ces données. Avant, de voir les tâches où l'Intelligence Artificielle peut intervenir, il est d'abord nécessaire de voir en détail la conception des ces IA.}
\section{L'intelligence artificielle}
\subsection{Définitions}
\textit{Qu'est ce qu'une intelligence artificielle ? Qu'est ce qu'un réseau de neurone ?}
\subsection{Les différents modèles existants}
\textit{Dans la vie suivant les informations qu'on possède, on ne traite pas cela de la même façons, il en va de même pour les réseaux de neurones. En effet, parfois on applique simplement quelque chose, d'autres fois on prend en compte nos expériences, et certaines fois notre attention se porte sur des objets particuliers. A la différence de notre grande capacité à traiter tout cela, l'IA, elle, a encore besoin de changer de réseaux de neurones faces à ces différentes situations. Respectivement, du réseaux de neurones "à propagation avant", à celui "récurrent", ou encore "séquentielle" (avec attention).}
\subsection{Le traitement automatisé du langage - Optionnel}
\textit{Quels sont les différents outils ?}\\
\textit{Suivant, comment les parties précédentes ont été traités, ou comment les parties futures seront discutées, cette partie pourrait ne pas être nécessaire. Sinon, elle regroupera l'idée de comment on passe de notre langue naturelle à celle de la machine, de passer aux mots à des vecteurs ? Quels traitements théoriques (théoriques pour ce distinguer de la pratique dans la partie future) doient être effectués sur les mots ? En fait cette partie fait référence aux chapitres 2 et 6 de Jurasky. Voir même le chapitre 9, en supprimant la sous partie précendente pour pouvoir parler directement des réseaux de neurones appliqués à la linguistique, en d'autres termes, des réseaux récurrents, des modèles séquentielles (encodeurs-décodeurs) avec l'attention, et des Transformers.}

\chapter{Les contributions de l'IA dans la linguistique historique}
\textit{C'est la partie 'Related Work', elle discute des différents aspects où la linguistique historique s'applique, à travers différents modèles.}
\section{Traitement de texte}
\subsection{Numérisation de textes historiques}
\subsection{Complétion de ces textes}
\section{Analyse de document}
\subsection{Extraction de données linguistiques}
\subsection{Identification de motifs}
\subsection{Classification de ces documents}
\section{Modélisation linguistique}
\subsection{Conception d'un modèle de langue}
\subsection{Simulation vers l'évolution d'une langue}

\chapter{Utilisation de l'IA pour la reconstruction de la proto-forme d'une langue au travers d'une expérience}
\textit{Ici, on se place dans un cas concret, pour montrer que ce n'est pas que de la théorie. En proposant une expérience.}
\section{Mise en place de l'expérience}
\subsection{Conceptualisation du problème}
\textit{Définir clairement le problème du titre, énoncer et justifier le choix de notre modèle réseaux de neurones et des différents outils appliqués. Voir s'il est possible de faire apparaitre plusieurs démarches, c'est à dire, une approche statistique et une approche neuronale (toujours pour renforcer et montrer le potentiel de l'IA).}
\subsection{Récupération de la base de données}
\textit{Il est fort possible que cette partie se regroupe avec la partie suivante, car il n'y aura pas grand chose à dire.}
\subsection{Potabiliser les données}
\textit{Le choix de "potabiliser", et non pas normaliser, est volontaire. En effet, la normalisation de nos données s'effectura dans un second temps dans les différentes démarches.}
\section{Démarche statistique}
\subsection{Méthode}
\subsection{Analyse}
\subsection{Critiques}
\textit{Il reste ici quelques sous parties à détailler.}
\section{Démarche neuronale}
\subsection{Méthode}
\subsection{Analyse}
\subsection{Critiques}
\textit{Il reste ici quelques sous parties à détailler.}

\chapter{Conclusion}
\section{Synthèse}
\textit{Résume tout ce qui a été dit.}
\section{Les différentes limites posées aujourd'hui}
\textit{une partie des limites aura déjà été traitée dans le chapitre précédent. Cette sous partie se veut résumer ces limites, et aller dans les limites générales (voir acutelles) de l'IA dans  la linguisitique historique.}
\section{Les perspectives de l'IA dans la linguistique historique}
\textit{Ouverture, dépassement de certaines limites, évolution des modèles.}

\chapter{Références}
\section{Bibliographie}


\end{document}